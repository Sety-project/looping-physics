\documentclass{article}
\usepackage{amsmath,amssymb,hyperref}

\begin{document}

% Remove paragraph indentation throughout the document
\setlength{\parindent}{0pt}

\title{Bootstrapping Looping Strategies}
\author{david.relkin@nomadic-labs.com}
\date{\today}
\maketitle

\begin{abstract}
    Although very profitable for users, looping strategies are resource-intensive and require skillful management at chain level.
    This paper presents a formalization of the bootstrapping optimization of a looping ecosystem, which allows to optimize the chain sponsor's resource consumption.

We only formalize the equilibrium state of the ecosystem, with strong but intuitive assumptions that lead to a closed form solution.
    
    The optimization problem is formulated as a constrained optimization problem and solved using the Lagrange multiplier method.
\end{abstract}

\section{Looping: a 4 Dapp Ecosystem}

Looping is made up of components:
    \begin{itemize}
        \item a yield-bearing asset with yield $r$, correlated to a base asset (eg mRe7 correlated to USDC).
        \item a Lending protocol (eg Gearbox passive pool):
        \begin{itemize}
            \item supply by users: $N_l$, and sponsor: $N_l^*$
            \item borrowed amount: $N_b<=(N_l+N_l^*)$
            \item the protocol's risk manager sets the borrow cap proportional to dex liquidity: $N_b<=g (N_d+N_d^*)$
            \item borrow rate curve: $r_b=IR(\frac{N_b}{N_l+N_l^*})$
            \item lending rate curve: $r_b(1-\epsilon)$
            \item reward\footnote{Rewards are defined as a fixed annual budget and the sponsor is blacklisted, ie $APY = \frac{R}{N}$} paid by sponsor: $R_l$
        \end{itemize}
        \item a DEX protocol (eg Curve)
        \begin{itemize}
            \item liquidity\footnote{DEX liquidity is defined as the sum of the two pools} supplied by users: $N_d$, and sponsor: $N_d^*$ 
            \item fee APY: $r_d\approx 0$
            \item reward paid by sponsor: $R_d$
        \end{itemize}
        \item a looping vault (eg Gearbox credit account)
        \begin{itemize}
            \item deposit by users only: $N_v$ (and $N_v^*=0)$
            \item We assume r is high enough so that loopers seek maximum leverage $l$, so that $N_b=N_v(l-1)$
            \item reward paid by sponsor: $R_v$
        \end{itemize}
    \end{itemize}

    \subsection{Borrowed amount: Lending + DEX constraint}
    The interest rate curve is parametrized by:
    \begin{equation*}
        IR(U) = \begin{cases}
        r_0+s_0 U & \text{if } U<=U_1\\
        r_1+s_1 (U-U_1) & \text{if } U_1<U<=U_2\\
        r_2+s_2 (U-U_2) & \text{if } U_2<U<=1
        \end{cases}
    \end{equation*}
    
    In a properly configured lending market, borrow rate naturally hovers around the kink\footnote{Gearbox has two kinks, we focus on the second kink}
    , which we set at a typical value of $90\%$.
    \footnote{See \href{https://morpho.org/blog/introducing-the-adaptivecurveirm-efficient-and-autonomous}{Morpho blog post}}.
    Assuming equilibrium is achieved at the kink greatly simplifies the problem: we can parametrize the IR curve only by $r_b(90\%)$. 
    \footnote{keeping a piecewise linear IR curve leads to a partitioned quadratic problem, which is still solvable but more complex}.
    \begin{equation*}
        \left\{
        \begin{aligned}
            U &= 0.9\\
            N_b &= U(N_l+N_l^*)\\
            r_b &= IR(U)
        \end{aligned}
        \right.
    \end{equation*}

    Also DEX liquidity is expensive for the sponsor, as it does not earn rewards but only small pool fees. 
    Hence it's optimal for sponsor to let DEX liquidity be just enough to cover the leverage loopers seek: 
    \begin{equation}
        \left\{
        \begin{aligned}
            N_v(l-1)&=\alpha (N_d+N_d^*), 
        \text{where } \alpha=\frac{1}{2}\\
        N_v(l-1)&=U(N_l+N_l^*),
        \text{where } U=0.9\\
        \end{aligned}
        \right.
        \label{eq:dex_liquidity_constraint}
    \end{equation}
    
    Sponsor liquidity ($N_l^*$, $N_d^*$) are thus determined by \ref{eq:dex_liquidity_constraint}.\\
    \begin{align*}
        N_l^* &= \frac{\beta_v R_v(l-1)}{U} - \beta_l R_l \\
        N_d^* &= \frac{\beta_v R_v(l-1)}{\alpha} - \beta_d R_d
    \end{align*}
    ($\beta$'s are defined in \ref{eq:user_tvl_relationships})
    
\subsection{Equilibrium APYs: empirical observations}
    Expressing each personna's APY:
    \begin{equation*}
    \left\{
    \begin{aligned}
        APY_l &= r_b(1-\epsilon) + \frac{R_l}{N_l}\\
        APY_d &\approx \frac{r}{2}+\frac{R_d}{N_d}\\
        APY_v &= l \space r - (l-1)r_b + \frac{R_v}{N_v}
    \end{aligned}
    \right.
    \end{equation*}
    
    Market APYs can be estimated by AB testing, ie shocking reward budget every epoch and observing where TVL stabilitizes 
    \footnote{In further studies we may estimate and use the half-life of this equilibrium and the present study dynamic}. \\
    For stable lending and DEX, we observed $APY_l=15\%$ and $APY_d=20\%$. \\
    For looping we assess the mimimum APY with high marketing impact to be $APY_v=75\%$.

    Given assumptions about user TVL's, we can now solve for user TVL:
    \begin{equation}
    \left\{
    \begin{aligned}
        N_l &= \beta_l R_l &\beta_l=\frac{1}{APY_l - r_b(1-\epsilon)}\\
        N_d &= \beta_d R_d &\beta_d=\frac{1}{APY_d - \frac{r}{2}}\\
        N_v &= \beta_v R_v &\beta_v=\frac{1}{APY_v - lr + (l-1)r_b}
    \end{aligned}
    \right.
    \label{eq:user_tvl_relationships}
    \end{equation}


\subsection{Sponsor budget}

The sponsor is bound by a liquidity budget and a reward target.
The latter includes reward paid, cost of capital (0 for now), offset by APY accrued:
\begin{equation*}
    \left\{
    \begin{aligned}
    N_l^* + N_d^* &<= N\\
    (R_l+R_d+R_v) - N_l^* r_b(1-\epsilon) - N_d^* \frac{r}{2} &= R
    \end{aligned}
    \right.
\end{equation*}
Also, all rewards and notional are positive.

\section{2D Linear Programming Problem}

Solving for $R_v$ using the budget constraint equality, the 2D linear programming problem becomes:

\begin{equation}
\max_{R_l, R_d} \quad \text{TVL} = \alpha_l R_l + \alpha_d R_d + \alpha_0
\end{equation}

Subject to the constraints:
    \begin{equation}
    \left\{
    \begin{aligned}
    \gamma_l R_l + \gamma_d R_d &\leq N + \gamma_0 \\
    R_l \geq 0 \\
    R_d \geq 0 \\
    \delta_l R_l + \delta_d R_d &\leq \delta_0 \\
    \epsilon_l R_l + \epsilon_d R_d &\leq \epsilon_0
    \end{aligned}
    \right.
    \end{equation}

where the constraint coefficients are defined using intermediate variables:

\begin{align}
D &= 1 - \frac{(l-1) r_b(1-\epsilon)}{(APY_v - lr + (l-1)r_b) U} - \frac{(l-1) r}{2\alpha (APY_v - lr + (l-1)r_b)} \\
A_l &= \frac{1}{APY_l - r_b(1-\epsilon)} \\
A_d &= \frac{1}{APY_d - \frac{r}{2}} \\
A_v &= \frac{1}{APY_v - lr + (l-1)r_b} \\
B_l &= 1 + \frac{r_b(1-\epsilon)}{APY_l - r_b(1-\epsilon)} \\
B_d &= 1 + \frac{\frac{r}{2}}{APY_d - \frac{r}{2}} \\
C &= (l-1)\left(\frac{1}{U} + \frac{1}{\alpha}\right)
\end{align}

The coefficients become:

\begin{align}
\alpha_l &= A_l + A_v \frac{B_l}{D} \\
\alpha_d &= A_d + A_v \frac{B_d}{D} \\
\alpha_0 &= A_v \frac{R}{D} \\
\gamma_l &= -A_l + A_v C \frac{B_l}{D} \\
\gamma_d &= -A_d + A_v C \frac{B_d}{D} \\
\gamma_0 &= A_v C \frac{R}{D} \\
\delta_l &= \frac{U(APY_v - lr + (l-1)r_b)}{(l-1)(APY_l - r_b(1-\epsilon))} - \frac{B_l}{D} \\
\delta_d &= -\frac{B_d}{D} \\
\delta_0 &= \frac{R}{D} \\
\epsilon_l &= -\frac{B_l}{D} \\
\epsilon_d &= \frac{\alpha(APY_v - lr + (l-1)r_b)}{(l-1)(APY_d - \frac{r}{2})} - \frac{B_d}{D} \\
\epsilon_0 &= \frac{R}{D}
\end{align}

\subsection{Summary}

The 2D linear programming problem has been reduced to:

\begin{itemize}
    \item \textbf{Decision variables}: $R_l$, $R_d$ (only 2 variables)
    \item \textbf{Objective function}: $\max \text{TVL} = \alpha_l R_l + \alpha_d R_d + \alpha_0$ (linear in $R_l$, $R_d$)
    \item \textbf{Constraints}: 4 linear inequality constraints in $R_l$, $R_d$
    \item \textbf{Solution method}: Standard 2D simplex algorithm by checking all feasible vertices
\end{itemize}

The variable $R_v$ has been eliminated as a parameter and is determined by the budget constraint equality. All constraints have been transformed into linear form with newly defined coefficients ($\alpha$, $\gamma$, $\delta$, $\epsilon$) that depend on the original $\beta$ coefficients and system parameters.

Note that positive user TVL directly constrains input parameters:
\begin{equation}
    \left\{
    \begin{aligned}
    \text{User vault:} \quad &APY_v - lr + (l-1)r_b \geq 0 \\
    \text{User lending:} \quad &APY_l - r_b(1-\epsilon) \geq 0 \\
    \text{User DEX:} \quad &APY_d - \frac{r}{2} \geq 0
    \end{aligned}
    \right.
\end{equation}

\end{document}


